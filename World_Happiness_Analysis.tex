\documentclass[]{article}
\usepackage{lmodern}
\usepackage{amssymb,amsmath}
\usepackage{ifxetex,ifluatex}
\usepackage{fixltx2e} % provides \textsubscript
\ifnum 0\ifxetex 1\fi\ifluatex 1\fi=0 % if pdftex
  \usepackage[T1]{fontenc}
  \usepackage[utf8]{inputenc}
\else % if luatex or xelatex
  \ifxetex
    \usepackage{mathspec}
  \else
    \usepackage{fontspec}
  \fi
  \defaultfontfeatures{Ligatures=TeX,Scale=MatchLowercase}
\fi
% use upquote if available, for straight quotes in verbatim environments
\IfFileExists{upquote.sty}{\usepackage{upquote}}{}
% use microtype if available
\IfFileExists{microtype.sty}{%
\usepackage{microtype}
\UseMicrotypeSet[protrusion]{basicmath} % disable protrusion for tt fonts
}{}
\usepackage[margin=1in]{geometry}
\usepackage{hyperref}
\hypersetup{unicode=true,
            pdftitle={World Happiness Analysis},
            pdfauthor={Mona Arami},
            pdfborder={0 0 0},
            breaklinks=true}
\urlstyle{same}  % don't use monospace font for urls
\usepackage{color}
\usepackage{fancyvrb}
\newcommand{\VerbBar}{|}
\newcommand{\VERB}{\Verb[commandchars=\\\{\}]}
\DefineVerbatimEnvironment{Highlighting}{Verbatim}{commandchars=\\\{\}}
% Add ',fontsize=\small' for more characters per line
\usepackage{framed}
\definecolor{shadecolor}{RGB}{248,248,248}
\newenvironment{Shaded}{\begin{snugshade}}{\end{snugshade}}
\newcommand{\AlertTok}[1]{\textcolor[rgb]{0.94,0.16,0.16}{#1}}
\newcommand{\AnnotationTok}[1]{\textcolor[rgb]{0.56,0.35,0.01}{\textbf{\textit{#1}}}}
\newcommand{\AttributeTok}[1]{\textcolor[rgb]{0.77,0.63,0.00}{#1}}
\newcommand{\BaseNTok}[1]{\textcolor[rgb]{0.00,0.00,0.81}{#1}}
\newcommand{\BuiltInTok}[1]{#1}
\newcommand{\CharTok}[1]{\textcolor[rgb]{0.31,0.60,0.02}{#1}}
\newcommand{\CommentTok}[1]{\textcolor[rgb]{0.56,0.35,0.01}{\textit{#1}}}
\newcommand{\CommentVarTok}[1]{\textcolor[rgb]{0.56,0.35,0.01}{\textbf{\textit{#1}}}}
\newcommand{\ConstantTok}[1]{\textcolor[rgb]{0.00,0.00,0.00}{#1}}
\newcommand{\ControlFlowTok}[1]{\textcolor[rgb]{0.13,0.29,0.53}{\textbf{#1}}}
\newcommand{\DataTypeTok}[1]{\textcolor[rgb]{0.13,0.29,0.53}{#1}}
\newcommand{\DecValTok}[1]{\textcolor[rgb]{0.00,0.00,0.81}{#1}}
\newcommand{\DocumentationTok}[1]{\textcolor[rgb]{0.56,0.35,0.01}{\textbf{\textit{#1}}}}
\newcommand{\ErrorTok}[1]{\textcolor[rgb]{0.64,0.00,0.00}{\textbf{#1}}}
\newcommand{\ExtensionTok}[1]{#1}
\newcommand{\FloatTok}[1]{\textcolor[rgb]{0.00,0.00,0.81}{#1}}
\newcommand{\FunctionTok}[1]{\textcolor[rgb]{0.00,0.00,0.00}{#1}}
\newcommand{\ImportTok}[1]{#1}
\newcommand{\InformationTok}[1]{\textcolor[rgb]{0.56,0.35,0.01}{\textbf{\textit{#1}}}}
\newcommand{\KeywordTok}[1]{\textcolor[rgb]{0.13,0.29,0.53}{\textbf{#1}}}
\newcommand{\NormalTok}[1]{#1}
\newcommand{\OperatorTok}[1]{\textcolor[rgb]{0.81,0.36,0.00}{\textbf{#1}}}
\newcommand{\OtherTok}[1]{\textcolor[rgb]{0.56,0.35,0.01}{#1}}
\newcommand{\PreprocessorTok}[1]{\textcolor[rgb]{0.56,0.35,0.01}{\textit{#1}}}
\newcommand{\RegionMarkerTok}[1]{#1}
\newcommand{\SpecialCharTok}[1]{\textcolor[rgb]{0.00,0.00,0.00}{#1}}
\newcommand{\SpecialStringTok}[1]{\textcolor[rgb]{0.31,0.60,0.02}{#1}}
\newcommand{\StringTok}[1]{\textcolor[rgb]{0.31,0.60,0.02}{#1}}
\newcommand{\VariableTok}[1]{\textcolor[rgb]{0.00,0.00,0.00}{#1}}
\newcommand{\VerbatimStringTok}[1]{\textcolor[rgb]{0.31,0.60,0.02}{#1}}
\newcommand{\WarningTok}[1]{\textcolor[rgb]{0.56,0.35,0.01}{\textbf{\textit{#1}}}}
\usepackage{longtable,booktabs}
\usepackage{graphicx,grffile}
\makeatletter
\def\maxwidth{\ifdim\Gin@nat@width>\linewidth\linewidth\else\Gin@nat@width\fi}
\def\maxheight{\ifdim\Gin@nat@height>\textheight\textheight\else\Gin@nat@height\fi}
\makeatother
% Scale images if necessary, so that they will not overflow the page
% margins by default, and it is still possible to overwrite the defaults
% using explicit options in \includegraphics[width, height, ...]{}
\setkeys{Gin}{width=\maxwidth,height=\maxheight,keepaspectratio}
\IfFileExists{parskip.sty}{%
\usepackage{parskip}
}{% else
\setlength{\parindent}{0pt}
\setlength{\parskip}{6pt plus 2pt minus 1pt}
}
\setlength{\emergencystretch}{3em}  % prevent overfull lines
\providecommand{\tightlist}{%
  \setlength{\itemsep}{0pt}\setlength{\parskip}{0pt}}
\setcounter{secnumdepth}{0}
% Redefines (sub)paragraphs to behave more like sections
\ifx\paragraph\undefined\else
\let\oldparagraph\paragraph
\renewcommand{\paragraph}[1]{\oldparagraph{#1}\mbox{}}
\fi
\ifx\subparagraph\undefined\else
\let\oldsubparagraph\subparagraph
\renewcommand{\subparagraph}[1]{\oldsubparagraph{#1}\mbox{}}
\fi

%%% Use protect on footnotes to avoid problems with footnotes in titles
\let\rmarkdownfootnote\footnote%
\def\footnote{\protect\rmarkdownfootnote}

%%% Change title format to be more compact
\usepackage{titling}

% Create subtitle command for use in maketitle
\providecommand{\subtitle}[1]{
  \posttitle{
    \begin{center}\large#1\end{center}
    }
}

\setlength{\droptitle}{-2em}

  \title{World Happiness Analysis}
    \pretitle{\vspace{\droptitle}\centering\huge}
  \posttitle{\par}
    \author{Mona Arami}
    \preauthor{\centering\large\emph}
  \postauthor{\par}
      \predate{\centering\large\emph}
  \postdate{\par}
    \date{June, 2019}


\begin{document}
\maketitle

\hypertarget{introduction}{%
\subsection{Introduction}\label{introduction}}

This is an analysis of the World Happiness Report from 2015-2017,
looking at worldwide and region-wise trends in happiness score as well
as patterns in the importance of the six factors of happiness in
determining overall happiness in each country. The data comes from the
Gallup World Poll.

The three datasets used in this analysis are available on
\href{https://www.kaggle.com/unsdsn/world-happiness/data}{Kaggle}.

\hypertarget{data-import-and-first-look}{%
\subsection{Data Import and First
Look}\label{data-import-and-first-look}}

Library imports:

\begin{Shaded}
\begin{Highlighting}[]
\KeywordTok{library}\NormalTok{(ggplot2)}
\KeywordTok{library}\NormalTok{(dplyr)}
\end{Highlighting}
\end{Shaded}

\begin{verbatim}
## 
## Attaching package: 'dplyr'
\end{verbatim}

\begin{verbatim}
## The following objects are masked from 'package:stats':
## 
##     filter, lag
\end{verbatim}

\begin{verbatim}
## The following objects are masked from 'package:base':
## 
##     intersect, setdiff, setequal, union
\end{verbatim}

\begin{Shaded}
\begin{Highlighting}[]
\KeywordTok{library}\NormalTok{(RColorBrewer)}
\KeywordTok{library}\NormalTok{(tidyr)}
\KeywordTok{library}\NormalTok{(knitr)}
\KeywordTok{library}\NormalTok{(readr)}
\end{Highlighting}
\end{Shaded}

\hypertarget{data-import-there-are-three-separate-datasets---2015-2016-and-2017.}{%
\subparagraph{Data import: There are three separate datasets - 2015,
2016, and
2017.}\label{data-import-there-are-three-separate-datasets---2015-2016-and-2017.}}

\begin{Shaded}
\begin{Highlighting}[]
\NormalTok{df15 <-}\StringTok{ }\KeywordTok{read_csv}\NormalTok{(}\StringTok{"2015.csv"}\NormalTok{)}
\end{Highlighting}
\end{Shaded}

\begin{verbatim}
## Parsed with column specification:
## cols(
##   Country = col_character(),
##   Region = col_character(),
##   `Happiness Rank` = col_double(),
##   `Happiness Score` = col_double(),
##   `Standard Error` = col_double(),
##   `Economy (GDP per Capita)` = col_double(),
##   Family = col_double(),
##   `Health (Life Expectancy)` = col_double(),
##   Freedom = col_double(),
##   `Trust (Government Corruption)` = col_double(),
##   Generosity = col_double(),
##   `Dystopia Residual` = col_double()
## )
\end{verbatim}

\begin{Shaded}
\begin{Highlighting}[]
\NormalTok{df16 <-}\StringTok{ }\KeywordTok{read_csv}\NormalTok{(}\StringTok{"2016.csv"}\NormalTok{)}
\end{Highlighting}
\end{Shaded}

\begin{verbatim}
## Parsed with column specification:
## cols(
##   Country = col_character(),
##   Region = col_character(),
##   `Happiness Rank` = col_double(),
##   `Happiness Score` = col_double(),
##   `Lower Confidence Interval` = col_double(),
##   `Upper Confidence Interval` = col_double(),
##   `Economy (GDP per Capita)` = col_double(),
##   Family = col_double(),
##   `Health (Life Expectancy)` = col_double(),
##   Freedom = col_double(),
##   `Trust (Government Corruption)` = col_double(),
##   Generosity = col_double(),
##   `Dystopia Residual` = col_double()
## )
\end{verbatim}

\begin{Shaded}
\begin{Highlighting}[]
\NormalTok{df17 <-}\StringTok{ }\KeywordTok{read_csv}\NormalTok{(}\StringTok{"2017.csv"}\NormalTok{)}
\end{Highlighting}
\end{Shaded}

\begin{verbatim}
## Parsed with column specification:
## cols(
##   Country = col_character(),
##   Happiness.Rank = col_double(),
##   Happiness.Score = col_double(),
##   Whisker.high = col_double(),
##   Whisker.low = col_double(),
##   Economy..GDP.per.Capita. = col_double(),
##   Family = col_double(),
##   Health..Life.Expectancy. = col_double(),
##   Freedom = col_double(),
##   Generosity = col_double(),
##   Trust..Government.Corruption. = col_double(),
##   Dystopia.Residual = col_double()
## )
\end{verbatim}

2015 includes 158 countries. 2016 includes 157 countries. 2017 includes
155 countries.

We're interested in the Country and Region columns of each dataset in
order to group overall happiness scores by country and world region.
(Note: 2017 does not have a region column, so we can create one by
joining the region data from another year's dataset by country ID and
filling in any missing cells manually). Beyond overall happiness score,
country, and region, we're also interested in the columns for Economy
(GDP per capita), Family, Health (Life Expectancy), Freedom, Trust
(Absence of Government Corruption), and Generosity. These are the six
factors of happiness included in the survey. The values in these columns
indicate how much each of the factors contributed to the overall
happiness score for each country, i.e.~how important they were.

Here's a peek at the relevant columns in the 2015 dataset:

\begin{Shaded}
\begin{Highlighting}[]
\NormalTok{df15 <-}\StringTok{ }\NormalTok{df15 }\OperatorTok\StringTok{ }\KeywordTok{select}\NormalTok{(Country, Region, }\StringTok{`}\DataTypeTok{Happiness Rank}\StringTok{`}\NormalTok{, }\StringTok{`}\DataTypeTok{Happiness Score}\StringTok{`}\NormalTok{, }\StringTok{`}\DataTypeTok{Economy (GDP per Capita)}\StringTok{`}\NormalTok{, Family, }\StringTok{`}\DataTypeTok{Health (Life Expectancy)}\StringTok{`}\NormalTok{, Freedom, }\StringTok{`}\DataTypeTok{Trust (Government Corruption)}\StringTok{`}\NormalTok{, Generosity)}
\KeywordTok{kable}\NormalTok{(}\KeywordTok{head}\NormalTok{(df15))}
\end{Highlighting}
\end{Shaded}

\begin{longtable}[]{@{}llrrrrrrrr@{}}
\toprule
Country & Region & Happiness Rank & Happiness Score & Economy (GDP per
Capita) & Family & Health (Life Expectancy) & Freedom & Trust
(Government Corruption) & Generosity\tabularnewline
\midrule
\endhead
Switzerland & Western Europe & 1 & 7.587 & 1.39651 & 1.34951 & 0.94143 &
0.66557 & 0.41978 & 0.29678\tabularnewline
Iceland & Western Europe & 2 & 7.561 & 1.30232 & 1.40223 & 0.94784 &
0.62877 & 0.14145 & 0.43630\tabularnewline
Denmark & Western Europe & 3 & 7.527 & 1.32548 & 1.36058 & 0.87464 &
0.64938 & 0.48357 & 0.34139\tabularnewline
Norway & Western Europe & 4 & 7.522 & 1.45900 & 1.33095 & 0.88521 &
0.66973 & 0.36503 & 0.34699\tabularnewline
Canada & North America & 5 & 7.427 & 1.32629 & 1.32261 & 0.90563 &
0.63297 & 0.32957 & 0.45811\tabularnewline
Finland & Western Europe & 6 & 7.406 & 1.29025 & 1.31826 & 0.88911 &
0.64169 & 0.41372 & 0.23351\tabularnewline
\bottomrule
\end{longtable}

\begin{Shaded}
\begin{Highlighting}[]
\NormalTok{df16 <-}\StringTok{ }\NormalTok{df16 }\OperatorTok\StringTok{ }\KeywordTok{select}\NormalTok{(Country, Region, }\StringTok{`}\DataTypeTok{Happiness Rank}\StringTok{`}\NormalTok{, }\StringTok{`}\DataTypeTok{Happiness Score}\StringTok{`}\NormalTok{, }\StringTok{`}\DataTypeTok{Economy (GDP per Capita)}\StringTok{`}\NormalTok{, Family, }\StringTok{`}\DataTypeTok{Health (Life Expectancy)}\StringTok{`}\NormalTok{, Freedom, }\StringTok{`}\DataTypeTok{Trust (Government Corruption)}\StringTok{`}\NormalTok{, Generosity)}
\KeywordTok{kable}\NormalTok{(}\KeywordTok{head}\NormalTok{(df16))}
\end{Highlighting}
\end{Shaded}

\begin{longtable}[]{@{}llrrrrrrrr@{}}
\toprule
Country & Region & Happiness Rank & Happiness Score & Economy (GDP per
Capita) & Family & Health (Life Expectancy) & Freedom & Trust
(Government Corruption) & Generosity\tabularnewline
\midrule
\endhead
Denmark & Western Europe & 1 & 7.526 & 1.44178 & 1.16374 & 0.79504 &
0.57941 & 0.44453 & 0.36171\tabularnewline
Switzerland & Western Europe & 2 & 7.509 & 1.52733 & 1.14524 & 0.86303 &
0.58557 & 0.41203 & 0.28083\tabularnewline
Iceland & Western Europe & 3 & 7.501 & 1.42666 & 1.18326 & 0.86733 &
0.56624 & 0.14975 & 0.47678\tabularnewline
Norway & Western Europe & 4 & 7.498 & 1.57744 & 1.12690 & 0.79579 &
0.59609 & 0.35776 & 0.37895\tabularnewline
Finland & Western Europe & 5 & 7.413 & 1.40598 & 1.13464 & 0.81091 &
0.57104 & 0.41004 & 0.25492\tabularnewline
Canada & North America & 6 & 7.404 & 1.44015 & 1.09610 & 0.82760 &
0.57370 & 0.31329 & 0.44834\tabularnewline
\bottomrule
\end{longtable}

\begin{Shaded}
\begin{Highlighting}[]
\NormalTok{df17 <-}\StringTok{ }\KeywordTok{left_join}\NormalTok{(df17, }\KeywordTok{select}\NormalTok{(df16, }\StringTok{"Country"}\NormalTok{, }\StringTok{"Region"}\NormalTok{), }\DataTypeTok{by =} \StringTok{"Country"}\NormalTok{)}
\NormalTok{df17}\OperatorTok{$}\NormalTok{Region[}\DecValTok{71}\NormalTok{] <-}\StringTok{ "Eastern Asia"}
\NormalTok{df17}\OperatorTok{$}\NormalTok{Region[}\DecValTok{33}\NormalTok{] <-}\StringTok{ "Eastern Asia"}
\NormalTok{df17}\OperatorTok{$}\NormalTok{Region[}\DecValTok{155}\NormalTok{] <-}\StringTok{ "Sub-Saharan Africa"}
\NormalTok{df17}\OperatorTok{$}\NormalTok{Region[}\DecValTok{113}\NormalTok{] <-}\StringTok{ "Sub-Saharan Africa"}
\NormalTok{df17}\OperatorTok{$}\NormalTok{Region[}\DecValTok{139}\NormalTok{] <-}\StringTok{ "Sub-Saharan Africa"}
\NormalTok{df17 <-}\StringTok{ }\NormalTok{df17 }\OperatorTok\StringTok{ }\KeywordTok{select}\NormalTok{(Country, Region, Happiness.Rank, Happiness.Score, Economy..GDP.per.Capita., Family, Health..Life.Expectancy., Freedom, Trust..Government.Corruption., Generosity)}
\KeywordTok{kable}\NormalTok{(}\KeywordTok{head}\NormalTok{(df17))}
\end{Highlighting}
\end{Shaded}

\begin{longtable}[]{@{}llrrrrrrrr@{}}
\toprule
Country & Region & Happiness.Rank & Happiness.Score &
Economy..GDP.per.Capita. & Family & Health..Life.Expectancy. & Freedom &
Trust..Government.Corruption. & Generosity\tabularnewline
\midrule
\endhead
Norway & Western Europe & 1 & 7.537 & 1.616463 & 1.533524 & 0.7966665 &
0.6354226 & 0.3159638 & 0.3620122\tabularnewline
Denmark & Western Europe & 2 & 7.522 & 1.482383 & 1.551122 & 0.7925655 &
0.6260067 & 0.4007701 & 0.3552805\tabularnewline
Iceland & Western Europe & 3 & 7.504 & 1.480633 & 1.610574 & 0.8335521 &
0.6271626 & 0.1535266 & 0.4755402\tabularnewline
Switzerland & Western Europe & 4 & 7.494 & 1.564980 & 1.516912 &
0.8581313 & 0.6200706 & 0.3670073 & 0.2905493\tabularnewline
Finland & Western Europe & 5 & 7.469 & 1.443572 & 1.540247 & 0.8091577 &
0.6179509 & 0.3826115 & 0.2454828\tabularnewline
Netherlands & Western Europe & 6 & 7.377 & 1.503945 & 1.428939 &
0.8106961 & 0.5853845 & 0.2826618 & 0.4704898\tabularnewline
It looks like & the rankings don' & t shift around ve & ry much for the
wo & rld's happiest countries. & & & & &\tabularnewline
\bottomrule
\end{longtable}

\hypertarget{data-analysis}{%
\subsection{Data Analysis}\label{data-analysis}}

Now for the fun stuff.

Here's a world map color-coded by each country's overall happiness score
in 2015.

\begin{Shaded}
\begin{Highlighting}[]
\NormalTok{worldmap <-}\StringTok{ }\KeywordTok{map_data}\NormalTok{(}\StringTok{"world"}\NormalTok{)}
\KeywordTok{names}\NormalTok{(worldmap)[}\KeywordTok{names}\NormalTok{(worldmap)}\OperatorTok{==}\StringTok{"region"}\NormalTok{] <-}\StringTok{ "Country"}
\NormalTok{worldmap}\OperatorTok{$}\NormalTok{Country[worldmap}\OperatorTok{$}\NormalTok{Country }\OperatorTok{==}\StringTok{ "USA"}\NormalTok{] <-}\StringTok{ "United States"}
\NormalTok{happy_world <-}\StringTok{ }\NormalTok{df15 }\OperatorTok
\StringTok{  }\KeywordTok{full_join}\NormalTok{(worldmap, }\DataTypeTok{by =} \StringTok{"Country"}\NormalTok{)}

\NormalTok{map_theme <-}\StringTok{ }\KeywordTok{theme}\NormalTok{(}
    \DataTypeTok{axis.title.x =} \KeywordTok{element_blank}\NormalTok{(),}
    \DataTypeTok{axis.text.x  =} \KeywordTok{element_blank}\NormalTok{(),}
    \DataTypeTok{axis.ticks.x =} \KeywordTok{element_blank}\NormalTok{(),}
    \DataTypeTok{axis.title.y =} \KeywordTok{element_blank}\NormalTok{(),}
    \DataTypeTok{axis.text.y  =} \KeywordTok{element_blank}\NormalTok{(),}
    \DataTypeTok{axis.ticks.y =} \KeywordTok{element_blank}\NormalTok{(),}
    \DataTypeTok{panel.background =} \KeywordTok{element_rect}\NormalTok{(}\DataTypeTok{fill =} \StringTok{"white"}\NormalTok{))}

\KeywordTok{ggplot}\NormalTok{(}\DataTypeTok{data =}\NormalTok{ happy_world, }\DataTypeTok{mapping =} \KeywordTok{aes}\NormalTok{(}\DataTypeTok{x =}\NormalTok{ long, }\DataTypeTok{y =}\NormalTok{ lat, }\DataTypeTok{group =}\NormalTok{ group)) }\OperatorTok{+}
\StringTok{  }\KeywordTok{geom_polygon}\NormalTok{(}\KeywordTok{aes}\NormalTok{(}\DataTypeTok{fill =} \StringTok{`}\DataTypeTok{Happiness Score}\StringTok{`}\NormalTok{))  }\OperatorTok{+}
\StringTok{  }\KeywordTok{scale_fill_continuous}\NormalTok{(}\DataTypeTok{low=}\StringTok{"thistle2"}\NormalTok{, }\DataTypeTok{high=}\StringTok{"darkred"}\NormalTok{, }\DataTypeTok{na.value=}\StringTok{"snow2"}\NormalTok{) }\OperatorTok{+}
\StringTok{  }\KeywordTok{coord_quickmap}\NormalTok{() }\OperatorTok{+}
\StringTok{  }\KeywordTok{labs}\NormalTok{(}\DataTypeTok{title =} \StringTok{"Happiness Around the World - 2015"}\NormalTok{) }\OperatorTok{+}
\StringTok{  }\NormalTok{map_theme}
\end{Highlighting}
\end{Shaded}

\includegraphics{World_Happiness_Analysis_files/figure-latex/unnamed-chunk-6-1.pdf}
The darker the red, the higher the happiness score. Regions in gray do
not have happiness data. The happiest regions of the world appear to be
in Europe, North and South America, Australia and New Zealand. Africa
appears to contain the lowest overall happiness scores.

Let's have a look at the average happiness of each world region for
2015.

\begin{Shaded}
\begin{Highlighting}[]
\NormalTok{dfavg <-}\StringTok{ }\NormalTok{df15 }\OperatorTok
\StringTok{  }\KeywordTok{select}\NormalTok{(Region, }\StringTok{`}\DataTypeTok{Happiness Score}\StringTok{`}\NormalTok{) }\OperatorTok
\StringTok{  }\KeywordTok{group_by}\NormalTok{(Region) }\OperatorTok
\StringTok{  }\KeywordTok{summarize}\NormalTok{(}\DataTypeTok{Average =} \KeywordTok{mean}\NormalTok{(}\StringTok{`}\DataTypeTok{Happiness Score}\StringTok{`}\NormalTok{)) }\OperatorTok
\StringTok{  }\KeywordTok{arrange}\NormalTok{(}\KeywordTok{desc}\NormalTok{(Average))}
\KeywordTok{kable}\NormalTok{(dfavg)}
\end{Highlighting}
\end{Shaded}

\begin{longtable}[]{@{}lr@{}}
\toprule
Region & Average\tabularnewline
\midrule
\endhead
Australia and New Zealand & 7.285000\tabularnewline
North America & 7.273000\tabularnewline
Western Europe & 6.689619\tabularnewline
Latin America and Caribbean & 6.144682\tabularnewline
Eastern Asia & 5.626167\tabularnewline
Middle East and Northern Africa & 5.406900\tabularnewline
Central and Eastern Europe & 5.332931\tabularnewline
Southeastern Asia & 5.317444\tabularnewline
Southern Asia & 4.580857\tabularnewline
Sub-Saharan Africa & 4.202800\tabularnewline
We can visualize these averages wi & th a boxplot:\tabularnewline
\bottomrule
\end{longtable}

\begin{Shaded}
\begin{Highlighting}[]
\KeywordTok{ggplot}\NormalTok{(}\DataTypeTok{data =}\NormalTok{ df15, }\KeywordTok{aes}\NormalTok{(}\DataTypeTok{x =}\NormalTok{ df15}\OperatorTok{$}\NormalTok{Region, }\DataTypeTok{y =}\NormalTok{ df15}\OperatorTok{$}\StringTok{`}\DataTypeTok{Happiness Score}\StringTok{`}\NormalTok{)) }\OperatorTok{+}
\StringTok{  }\KeywordTok{geom_boxplot}\NormalTok{(}\KeywordTok{aes}\NormalTok{(}\DataTypeTok{color =}\NormalTok{ Region, }\DataTypeTok{fill =}\NormalTok{ Region), }\DataTypeTok{alpha =} \FloatTok{0.5}\NormalTok{) }\OperatorTok{+}
\StringTok{  }\KeywordTok{geom_point}\NormalTok{(}\KeywordTok{aes}\NormalTok{(}\DataTypeTok{color =}\NormalTok{ Region), }\DataTypeTok{position =} \KeywordTok{position_jitter}\NormalTok{(}\DataTypeTok{width =} \FloatTok{.1}\NormalTok{)) }\OperatorTok{+}
\StringTok{  }\KeywordTok{labs}\NormalTok{(}\DataTypeTok{title =} \StringTok{"Happiness by World Region - 2015"}\NormalTok{, }
       \DataTypeTok{x =} \StringTok{"Region"}\NormalTok{, }
       \DataTypeTok{y =} \StringTok{"Happiness Score"}\NormalTok{) }\OperatorTok{+}
\StringTok{  }\KeywordTok{theme_minimal}\NormalTok{() }\OperatorTok{+}
\StringTok{  }\KeywordTok{theme}\NormalTok{(}\DataTypeTok{plot.title =} \KeywordTok{element_text}\NormalTok{(}\DataTypeTok{size =} \KeywordTok{rel}\NormalTok{(}\FloatTok{2.5}\NormalTok{)),}
        \DataTypeTok{axis.title =} \KeywordTok{element_text}\NormalTok{(}\DataTypeTok{size =} \KeywordTok{rel}\NormalTok{(}\FloatTok{1.5}\NormalTok{)),}
        \DataTypeTok{axis.text.x =} \KeywordTok{element_blank}\NormalTok{())}
\end{Highlighting}
\end{Shaded}

\includegraphics{World_Happiness_Analysis_files/figure-latex/unnamed-chunk-8-1.pdf}
The table of averages and the boxplot both confirm the intuitions we had
about the world map for 2015. Australia \& New Zealand, North America,
and Western Europe have the highest average happiness scores.
Sub-Saharan Africa has the lowest average, next to Southern Asia.

We can make similar boxplots for 2016 and 2017.

\begin{Shaded}
\begin{Highlighting}[]
\KeywordTok{ggplot}\NormalTok{(}\DataTypeTok{data =}\NormalTok{ df16, }\KeywordTok{aes}\NormalTok{(}\DataTypeTok{x =}\NormalTok{ df16}\OperatorTok{$}\NormalTok{Region, }\DataTypeTok{y =}\NormalTok{ df16}\OperatorTok{$}\StringTok{`}\DataTypeTok{Happiness Score}\StringTok{`}\NormalTok{)) }\OperatorTok{+}
\StringTok{  }\KeywordTok{geom_boxplot}\NormalTok{(}\KeywordTok{aes}\NormalTok{(}\DataTypeTok{color =}\NormalTok{ Region, }\DataTypeTok{fill =}\NormalTok{ Region), }\DataTypeTok{alpha =} \FloatTok{0.5}\NormalTok{) }\OperatorTok{+}
\StringTok{  }\KeywordTok{geom_point}\NormalTok{(}\KeywordTok{aes}\NormalTok{(}\DataTypeTok{color =}\NormalTok{ Region), }\DataTypeTok{position =} \KeywordTok{position_jitter}\NormalTok{(}\DataTypeTok{width =} \FloatTok{.1}\NormalTok{)) }\OperatorTok{+}
\StringTok{  }\KeywordTok{labs}\NormalTok{(}\DataTypeTok{title =} \StringTok{"Happiness by World Region - 2016"}\NormalTok{, }
       \DataTypeTok{x =} \StringTok{"Region"}\NormalTok{, }
       \DataTypeTok{y =} \StringTok{"Happiness Score"}\NormalTok{) }\OperatorTok{+}
\StringTok{  }\KeywordTok{theme_minimal}\NormalTok{() }\OperatorTok{+}
\StringTok{  }\KeywordTok{theme}\NormalTok{(}\DataTypeTok{plot.title =} \KeywordTok{element_text}\NormalTok{(}\DataTypeTok{size =} \KeywordTok{rel}\NormalTok{(}\FloatTok{2.5}\NormalTok{)),}
        \DataTypeTok{axis.title =} \KeywordTok{element_text}\NormalTok{(}\DataTypeTok{size =} \KeywordTok{rel}\NormalTok{(}\FloatTok{1.5}\NormalTok{)),}
        \DataTypeTok{axis.text.x =} \KeywordTok{element_blank}\NormalTok{())}
\end{Highlighting}
\end{Shaded}

\includegraphics{World_Happiness_Analysis_files/figure-latex/unnamed-chunk-9-1.pdf}

\begin{Shaded}
\begin{Highlighting}[]
\KeywordTok{ggplot}\NormalTok{(}\DataTypeTok{data =}\NormalTok{ df17, }\KeywordTok{aes}\NormalTok{(}\DataTypeTok{x =}\NormalTok{ df17}\OperatorTok{$}\NormalTok{Region, }\DataTypeTok{y =}\NormalTok{ df17}\OperatorTok{$}\NormalTok{Happiness.Score)) }\OperatorTok{+}
\StringTok{  }\KeywordTok{geom_boxplot}\NormalTok{(}\KeywordTok{aes}\NormalTok{(}\DataTypeTok{color =}\NormalTok{ Region, }\DataTypeTok{fill =}\NormalTok{ Region), }\DataTypeTok{alpha =} \FloatTok{0.5}\NormalTok{) }\OperatorTok{+}
\StringTok{  }\KeywordTok{geom_point}\NormalTok{(}\KeywordTok{aes}\NormalTok{(}\DataTypeTok{color =}\NormalTok{ Region), }\DataTypeTok{position =} \KeywordTok{position_jitter}\NormalTok{(}\DataTypeTok{width =} \FloatTok{.1}\NormalTok{)) }\OperatorTok{+}
\StringTok{  }\KeywordTok{labs}\NormalTok{(}\DataTypeTok{title =} \StringTok{"Happiness by World Region - 2017"}\NormalTok{, }
       \DataTypeTok{x =} \StringTok{"Region"}\NormalTok{, }
       \DataTypeTok{y =} \StringTok{"Happiness Score"}\NormalTok{) }\OperatorTok{+}
\StringTok{  }\KeywordTok{theme_minimal}\NormalTok{() }\OperatorTok{+}
\StringTok{  }\KeywordTok{theme}\NormalTok{(}\DataTypeTok{plot.title =} \KeywordTok{element_text}\NormalTok{(}\DataTypeTok{size =} \KeywordTok{rel}\NormalTok{(}\FloatTok{2.5}\NormalTok{)),}
        \DataTypeTok{axis.title =} \KeywordTok{element_text}\NormalTok{(}\DataTypeTok{size =} \KeywordTok{rel}\NormalTok{(}\FloatTok{1.5}\NormalTok{)),}
        \DataTypeTok{axis.text.x =} \KeywordTok{element_blank}\NormalTok{())}
\end{Highlighting}
\end{Shaded}

\includegraphics{World_Happiness_Analysis_files/figure-latex/unnamed-chunk-10-1.pdf}
Just from these boxplots, we can tell that the average happiness scores
across world regions don't change very much from 2015-2017.

Let's look at how happiness scores changed for each country over time.

\begin{Shaded}
\begin{Highlighting}[]
\NormalTok{df15}\OperatorTok{$}\NormalTok{year <-}\StringTok{ "2015"}
\NormalTok{df16}\OperatorTok{$}\NormalTok{year <-}\StringTok{ "2016"}
\NormalTok{df17}\OperatorTok{$}\NormalTok{year <-}\StringTok{ "2017"}

\KeywordTok{names}\NormalTok{(df15)[}\KeywordTok{names}\NormalTok{(df15)}\OperatorTok{==}\StringTok{"Happiness Score"}\NormalTok{] <-}\StringTok{ "score"}
\KeywordTok{names}\NormalTok{(df16)[}\KeywordTok{names}\NormalTok{(df16)}\OperatorTok{==}\StringTok{"Happiness Score"}\NormalTok{] <-}\StringTok{ "score"}
\KeywordTok{names}\NormalTok{(df17)[}\KeywordTok{names}\NormalTok{(df17)}\OperatorTok{==}\StringTok{"Happiness.Score"}\NormalTok{] <-}\StringTok{ "score"}

\NormalTok{dfall <-}\StringTok{ }\KeywordTok{rbind}\NormalTok{(}\KeywordTok{select}\NormalTok{(df15,}\StringTok{"Country"}\NormalTok{, }\StringTok{"Region"}\NormalTok{, }\StringTok{"score"}\NormalTok{, }\StringTok{"year"}\NormalTok{),}
               \KeywordTok{select}\NormalTok{(df16, }\StringTok{"Country"}\NormalTok{, }\StringTok{"Region"}\NormalTok{, }\StringTok{"score"}\NormalTok{, }\StringTok{"year"}\NormalTok{),}
               \KeywordTok{select}\NormalTok{(df17, }\StringTok{"Country"}\NormalTok{, }\StringTok{"Region"}\NormalTok{, }\StringTok{"score"}\NormalTok{, }\StringTok{"year"}\NormalTok{))}

\KeywordTok{ggplot}\NormalTok{(}\DataTypeTok{data =}\NormalTok{ dfall) }\OperatorTok{+}
\StringTok{  }\KeywordTok{geom_line}\NormalTok{(}\DataTypeTok{mapping =} \KeywordTok{aes}\NormalTok{(}\DataTypeTok{x =}\NormalTok{ year, }\DataTypeTok{y =}\NormalTok{ score, }\DataTypeTok{group =}\NormalTok{ Country, }
                          \DataTypeTok{color =}\NormalTok{ Region),}
            \DataTypeTok{alpha =} \FloatTok{0.5}\NormalTok{, }\DataTypeTok{show.legend =} \OtherTok{FALSE}\NormalTok{) }\OperatorTok{+}
\StringTok{  }\KeywordTok{geom_point}\NormalTok{(}\KeywordTok{aes}\NormalTok{(}\DataTypeTok{x =}\NormalTok{ year, }\DataTypeTok{y =}\NormalTok{ score, }\DataTypeTok{color =}\NormalTok{ Region), }
             \DataTypeTok{position =} \KeywordTok{position_jitter}\NormalTok{(}\DataTypeTok{width =} \FloatTok{.1}\NormalTok{),}
             \DataTypeTok{alpha =} \FloatTok{0.5}\NormalTok{,}
             \DataTypeTok{show.legend =} \OtherTok{FALSE}\NormalTok{) }\OperatorTok{+}
\StringTok{  }\KeywordTok{labs}\NormalTok{(}\DataTypeTok{title =} \StringTok{"Worldwide Happiness Scores 2015-17"}\NormalTok{, }
       \DataTypeTok{x =} \StringTok{"Year"}\NormalTok{, }
       \DataTypeTok{y =} \StringTok{"Happiness Score"}\NormalTok{) }\OperatorTok{+}
\StringTok{  }\KeywordTok{theme_minimal}\NormalTok{() }\OperatorTok{+}
\StringTok{  }\KeywordTok{theme}\NormalTok{(}\DataTypeTok{plot.title =} \KeywordTok{element_text}\NormalTok{(}\DataTypeTok{size =} \KeywordTok{rel}\NormalTok{(}\FloatTok{2.5}\NormalTok{)),}
        \DataTypeTok{axis.title =} \KeywordTok{element_text}\NormalTok{(}\DataTypeTok{size =} \KeywordTok{rel}\NormalTok{(}\FloatTok{1.5}\NormalTok{)),}
        \DataTypeTok{strip.text.x =} \KeywordTok{element_text}\NormalTok{(}\DataTypeTok{size =} \KeywordTok{rel}\NormalTok{(}\FloatTok{1.5}\NormalTok{))) }\OperatorTok{+}
\StringTok{  }\KeywordTok{facet_wrap}\NormalTok{(}\OperatorTok{~}\StringTok{ }\NormalTok{Region)}
\end{Highlighting}
\end{Shaded}

\includegraphics{World_Happiness_Analysis_files/figure-latex/unnamed-chunk-11-1.pdf}
For the most part, the scores for each country do not change
significantly from 2015-2017. There are very few countries whose scores
decreased significantly, and fewer still whose scores increased
significantly. The countries that underwent significant change, if any,
were primarily in Sub-Saharan Africa or Latin America \& the Caribbean.
This makes sense, since countries in these regions are more subject to
sudden changes in economy and political stability.

To explore the factors that could be contributing to the score
differences between world regions, let's have a look at the six factors
of happiness for each of these regions. We'll use the 2015 data.

\begin{Shaded}
\begin{Highlighting}[]
\CommentTok{# 2015}
\KeywordTok{names}\NormalTok{(df15)[}\KeywordTok{names}\NormalTok{(df15)}\OperatorTok{==}\StringTok{"Economy (GDP per Capita)"}\NormalTok{] <-}\StringTok{ "Economy"}
\KeywordTok{names}\NormalTok{(df15)[}\KeywordTok{names}\NormalTok{(df15)}\OperatorTok{==}\StringTok{"Health (Life Expectancy)"}\NormalTok{] <-}\StringTok{ "Health"}
\KeywordTok{names}\NormalTok{(df15)[}\KeywordTok{names}\NormalTok{(df15)}\OperatorTok{==}\StringTok{"Trust (Government Corruption)"}\NormalTok{] <-}\StringTok{ "Trust"}
\KeywordTok{pairs}\NormalTok{(}\OperatorTok{~}\StringTok{ }\NormalTok{Economy}\OperatorTok{+}\NormalTok{Family}\OperatorTok{+}\NormalTok{Health}\OperatorTok{+}\NormalTok{Freedom}\OperatorTok{+}\NormalTok{Trust}\OperatorTok{+}\NormalTok{Generosity, }\DataTypeTok{data =}\NormalTok{ df15, }
      \DataTypeTok{main=}\StringTok{"Importances of the Six Factors of Happiness"}\NormalTok{)}
\end{Highlighting}
\end{Shaded}

\includegraphics{World_Happiness_Analysis_files/figure-latex/unnamed-chunk-12-1.pdf}

This pairs plot compares the importance of each of the six factors of
happiness to each of the others. If there is a strong positive linear
correlation between two factors, we can say that if one factor is
important in evaluating a country's overall happiness, it is likely that
the other factor is important as well. Based on the plots, it seems that
the importances of Economy \& Health are strongly correlated, as well as
Economy \& Family.

I was hoping to be able to draw regression lines over each of the pair
plots, but couldn't figure out how. I've settled with eyeballing the
correlations.

Let's take a closer look at the top 10 happiest countries in 2015 and
how much each of the six factors contributed toward their overall
happiness scores. For this, a stacked bar plot would be a useful
visualization.

\begin{Shaded}
\begin{Highlighting}[]
\NormalTok{dfwide <-}\StringTok{ }\NormalTok{df15 }\OperatorTok
\StringTok{  }\KeywordTok{head}\NormalTok{(}\DecValTok{10}\NormalTok{)}

\NormalTok{dflong <-}\StringTok{ }\KeywordTok{gather}\NormalTok{(dfwide, Factor, }\StringTok{`}\DataTypeTok{Importance of Factor}\StringTok{`}\NormalTok{, Economy}\OperatorTok{:}\NormalTok{Generosity, }\DataTypeTok{factor_key=}\OtherTok{TRUE}\NormalTok{)}

\KeywordTok{ggplot}\NormalTok{(}\DataTypeTok{data =}\NormalTok{ dflong) }\OperatorTok{+}
\StringTok{  }\KeywordTok{geom_bar}\NormalTok{(}\DataTypeTok{stat =} \StringTok{"identity"}\NormalTok{, }
           \KeywordTok{aes}\NormalTok{(}\DataTypeTok{x =}\NormalTok{ Country, }\DataTypeTok{y =} \StringTok{`}\DataTypeTok{Importance of Factor}\StringTok{`}\NormalTok{, }\DataTypeTok{fill =}\NormalTok{ Factor)) }\OperatorTok{+}
\StringTok{  }\KeywordTok{coord_flip}\NormalTok{() }\OperatorTok{+}
\StringTok{  }\KeywordTok{theme_minimal}\NormalTok{() }\OperatorTok{+}
\StringTok{  }\KeywordTok{theme}\NormalTok{(}\DataTypeTok{legend.position =} \StringTok{"top"}\NormalTok{) }\OperatorTok{+}
\StringTok{  }\KeywordTok{labs}\NormalTok{(}\DataTypeTok{title =} \StringTok{"The Six Factors of Happiness in the Ten Happiest Countries"}\NormalTok{) }\OperatorTok{+}
\StringTok{  }\KeywordTok{theme}\NormalTok{(}\DataTypeTok{plot.title =} \KeywordTok{element_text}\NormalTok{(}\DataTypeTok{size =} \KeywordTok{rel}\NormalTok{(}\FloatTok{1.5}\NormalTok{)),}
        \DataTypeTok{axis.title =} \KeywordTok{element_text}\NormalTok{(}\DataTypeTok{size =} \KeywordTok{rel}\NormalTok{(}\FloatTok{1.5}\NormalTok{)))}
\end{Highlighting}
\end{Shaded}

\includegraphics{World_Happiness_Analysis_files/figure-latex/unnamed-chunk-13-1.pdf}
In general, Economy and Family seem to the the two most important
factors of happiness in these countries. Trust (absence of corruption)
and Generosity are the least important.

These six factors don't add up to the overall happiness score for each
country in the bar plot because they should just be thought of as
weights and the `Dystopian Residual' isn't taken into account. More
information on what the `Dystopian Residual' is can be found in the data
overview on Kaggle

\hypertarget{conclusion}{%
\subsection{Conclusion}\label{conclusion}}

This analysis illustrated that the world's happiest countries are
primarily in Western Europe (especially Northern Europe), North America,
and Australia \& New Zealand. These averages did not change very much
from 2015-2017. It also revealed that Economy (GDP per capita) is the
most important factor in evaluating a country's happiness.
Unsurprisingly, the happiest countries and world regions generally
tended to be ones with strong and stable economies. The importance of
Economy is also strongly positively correlated with those of Family and
Health. This is expected, since more economic stability and higher GDP
per capita generally encourages stable and comfortable family life as
well as increases the availability of proper medical resources and
healthcare. These factors then weigh more when determining overall
happiness.

I would also hypothesize that these three factors--Economy, Family, and
Health--tend to be particularly important because they directly affect
individuals living in these countries. Everyone is affected by the state
of the economy, especially since it holds direct sway over the
availability and security of jobs and the flow of money. Families are
the nucleus of home life for most individuals, and Health also affects
people on the level of individuals. Consequently, these are very
concrete factors and therefore have more influence on the happiness
score gauged by individuals.

Sub-Saharan Africa and Southern Asia could definitely use a lift, but
overall, the world doesn't seem to be doing too badly. Here's to the
future!


\end{document}
